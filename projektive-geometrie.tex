\chapter{Projektive Geometrie}
\section{Hessesche Normalform}

\section{Skalare und Vektoren und das Kartesisches Koordinatensystem}

\begin{description}
	\item[Skalar]
	ist eine reelle (oder komplexe) Zahl. Beispiele: Temperatur,
	Druck, Luftfeuchtigkeit.
	\item[Vektor]
	hat einen (reellen) Betrag und eine Richtung. Beispiele:
	(Wind-) Geschwindigkeit (an einem Ort), Kraft (auf ein
	Objekt), Fliessgeschwindigkeit in Gewässern, Elektrisches
	Feld, Magnetfeld, Gravitationsfeld, etc.
\end{description}

\subsection{Addition von Vektoren und Multiplikation mit einem Skalar}

Gegeben sind die Vektoren:
\begin{math}
	\vec{a} = 
	\begin{pmatrix} a_{1} \\ a_{2} \end{pmatrix}
\end{math}
und
\begin{math}
	\vec{b} = 
	\begin{pmatrix} b_{1} \\ b_{2} \end{pmatrix}
\end{math}

\begin{description}
	\item[Addition:]
	Zwei Vektoren $\vec{a}$ und $\vec{b}$ addieren heisst entsprechende Komponenten addieren. 
\end{description}

\begin{math}
	\vec{a} + \vec{b} = 
	\begin{pmatrix} a_{1} \\ a_{2} \end{pmatrix} +
	\begin{pmatrix} b_{1} \\ b_{2} \end{pmatrix} =
	\begin{pmatrix} a_{1} + b_{1} \\ a_{2} + b_{2} \end{pmatrix}
\end{math}

\begin{description}
	\item[Multiplikation mit Skalar:]
	Einen Vektor $\vec{a}$ mit einem Skalar $\lambda \in \mathbb{R}$ multiplizieren. 
\end{description}

\begin{math}
	\lambda \vec{a} = 
	\lambda \begin{pmatrix} a_{1} \\ a_{2} \end{pmatrix} =
	\begin{pmatrix} \lambda a_{1} \\ \lambda a_{2} \end{pmatrix}
\end{math}

\subsection{Das Inverse eines Vektors und der Nullvektor}

\begin{description}
	\item[Inverse:]
	Das Inverse $-\vec{a}$ des Vektors
	\begin{math}
		\vec{a} = \begin{pmatrix} a_{1} \\ a_{2} \end{pmatrix}
	\end{math}
	ist der Vektor mit den negativen Komponenten:
	\begin{math}
		-\vec{a} = -\begin{pmatrix} a_{1} \\ a_{2} \end{pmatrix} =
		\begin{pmatrix} -a_{1} \\ -a_{2} \end{pmatrix}
	\end{math}
\end{description}

\begin{description}
	\item[Nullvektor:]
	Der Nullvektor $\vec{0}$ ist ein Vektor dessen Komponenten alle
	verschwinden (also gleich Null sind):
	\begin{math}
		\vec{0} = \begin{pmatrix} 0 \\ 0 \end{pmatrix}
	\end{math}
\end{description}

Damit wird die Subtraktion des Vektors $\vec{b}$ vom Vektor $\vec{b}$ wie folgt definiert:
\begin{math}
	\vec{a} -\vec{b} = \vec{a} + \begin{pmatrix} -\vec{b} \end{pmatrix}
\end{math}

\begin{description}
	\item[Rechenregeln:]
\end{description}
\begin{math}
	\vec{a} + \vec{b} = \vec{b} + \vec{a}
\end{math}
Kommutativgesetz\\
\begin{math}
	\vec{a} + (\vec{b} + \vec{c}) = (\vec{a} + \vec{b}) + \vec{c}
\end{math}
Assoziativgesetz\\
\begin{math}
	\vec{a} + 0 = \vec{a}
\end{math}
Existenz eines Neutralelements $\vec{0}$\\
$\vec{a} + \vec{-a} = \vec{0}$\\
$\lambda (\vec{a} + \vec{b}) = \lambda \vec{a} + \lambda \vec{b}$\\
$(\lambda + \mu) \vec{a} = \lambda \vec{a} + \mu \vec{a}$\\
$(\lambda \mu) \vec{a} = \lambda (\mu \vec{a}) = \mu (\lambda \vec{a})$
\newpage

\subsection{Geometrische Interpretation}

Zwei Vektoren sind gleich, wenn ihre Komponenten gleich sind!
Achtung: In der Physik darf man z.B. Kraftvektoren nicht einfach
verschieben!

\begin{figure}[!ht]
	\centering
	\includegraphics[width=0.7\linewidth]{fig/geometrische_interpretation}
	\caption{Geometrische Interpretation}
	\label{fig:geometrische_interpretation}
\end{figure}

Es folgt nun ein Example zur Berechnung von Vektoren.
Bestimmen Sie
\begin{math}
	3 \vec{a} - 2 \vec{b} + \vec{c}
\end{math}
sowohl grafisch wie auch rechnerisch (analytisch). Details der Vektoren sind in Abbildung \ref{fig:geometrische_interpretation} zu entnehmen.

\begin{math}
	3 \begin{pmatrix} 4 \\ -2 \end{pmatrix} - 
	2 \begin{pmatrix} -1 \\ 4 \end{pmatrix} + 
	\begin{pmatrix} -3 \\ -4 \end{pmatrix} = 
	\begin{pmatrix} 12 \\ -6 \end{pmatrix} -
	\begin{pmatrix} -2 \\ 8 \end{pmatrix} +
	\begin{pmatrix} -3 \\ -4 \end{pmatrix} =
	\begin{pmatrix} 11 \\ -18 \end{pmatrix}
\end{math}

\begin{description}
	\item[Basisvektoren:]
	Die Basisvektoren $\vec{e}_x$ und $\vec{e}_y$ sind orthogonal und haben die Länge 1,\\
	d.h. $\vec{e}_x \bullet \vec{e}_x = 0$ und $\mid \vec{e}_x \mid = 1$ sowie $\mid \vec{e}_y \mid = 1$.
\end{description}

\section{Skalarprodukt}

Das Skalarprodukt zweier Vektoren $\vec{a}$ und $\vec{b}$ ist wie folgt definiert:\\
$\vec{a} \bullet \vec{b} = \mid \vec{a} \mid \dot \mid \vec{b} \mid cos(\phi)$\\
In kartesischen Koordinaten gilt:\\
$\vec{a} \bullet \vec{b} =
\begin{pmatrix} a_{1} \\ a_{2} \end{pmatrix} \bullet
\begin{pmatrix} b_{1} \\ b_{2} \end{pmatrix} =
a_{1} b_{1} + a_{2} b_{2}$

Es folgt ein Beispiel dazu:\\
Gegeben sind die Vektoren 
$\vec{a} = \begin{pmatrix} 4 \\ -2 \end{pmatrix}$
und
$\vec{b} = \begin{pmatrix} -1 \\ 4 \end{pmatrix}$
man berechne nun das Skalarprodukt sowie den Winkel $\phi$.\\

\begin{math}
	\vec{a} \bullet \vec{b} = 
	\begin{pmatrix} 4 \\ -2 \end{pmatrix} + 
	\begin{pmatrix} -1 \\ 4 \end{pmatrix} =
	4 (-2) + (-1) 4 = -4 -8 = -12
\end{math}

\begin{math}
	a = \mid \vec{a} \mid = 
	\sqrt{\vec{a_{1}^2} + \vec{a_{2}^2}} =
	\sqrt{4^2 + (-2)^2} =
	\sqrt{20}
\end{math}

\begin{math}
	b = \mid \vec{b} \mid = 
	\sqrt{\vec{b_{1}^2} + \vec{b_{2}^2}} =
	\sqrt{(-1)^2 + 4^2} =
	\sqrt{17}
\end{math}

\begin{math}
	cos(\phi) =
	\frac{\vec{a} \bullet \vec{b}}{\mid \vec{a} \mid \dot \mid \vec{b} \mid} = 
	\frac{-12}{\sqrt{20} \sqrt{17}} = 
	-0.6508
\end{math}
\\Daraus folgt mit $\arccos(\phi)$:
\begin{math}
	\phi = \arccos(-0.6508) = 130.6^\circ
\end{math}
\\$\mid \vec{a} \mid$ ist die Länge des Vectors $\vec{a}$.

\begin{description}
	\item[Rechengesetze]
\end{description}
$\vec{a} \bullet \vec{b} = \vec{b} \bullet \vec{a}$ Kommutativgesetz\\
$\vec{a} \bullet (\vec{b} + \vec{c}) = 
\vec{a} \bullet \vec{b} + \vec{a} \bullet \vec{c}$
Distributivgesetz\\
$\lambda (\vec{a} \bullet \vec{b}) = 
(\lambda \vec{a}) \bullet \vec{b} = 
\vec{a} \bullet (\lambda \vec{b})$

\begin{description}
	\item[Orthogonale Vektoren]
	Zwei Vektoren $\vec{a}$ und $\vec{b}$ stehen genau dann senkrecht aufeinander, sind also orthogonal, falls ihr Skalarprodukt verschwindet respektive 0 ist.\\
	\begin{math}
		\vec{a} \bullet \vec{b} = 0 \iff \vec{a} \perp \vec{b}
	\end{math}
	\\Angenommen ein der Vektor $\vec{a} = \begin{pmatrix} 2 \\ 1 \end{pmatrix}$ sei gegeben. Um einen dazugehörenden orthogonalen Vektor zu erhalten muss folgende Formel aufgelöst werden:\\
	\begin{math}
		2b_{1} + 1b_2 = 0
	\end{math}
	\\Ein möglicher Vektor wäre also $\vec{b} = \begin{pmatrix} -1 \\ 2 \end{pmatrix}$ oder ein Vielfaches davon!
\end{description}
\section{Spatprodukt}

Das Spatprodukt $[\vec{a}, \vec{b}, \vec{c}]$ der drei Vektoren $\vec{a}$, $\vec{b}$ und $\vec{c}$ ist das Skalar $[\vec{a}, \vec{b}, \vec{c}] = \vec{a} \bullet (\vec{b} \times \vec{c})$.
Der Betrag des Spatprodukts $\mid [\vec{a}, \vec{b}, \vec{c}] \mid$ ist das Volumen des durch die drei Vektoren $\vec{a}$, $\vec{b}$ und $\vec{c}$ aufgespannten Spats.

\begin{figure}[!ht]
	\centering
	\includegraphics[width=0.7\linewidth]{fig/spatprodukt}
	\caption{Spatprodukt}
	\label{fig:spatprodukt}
\end{figure}

\begin{description}
	\item[Rechenregeln für das Spatprodukt]
\end{description}
Vertauschen von zwei Vektoren bewirkt einen Vorzeichenwechsel: z.B.\\
$[\vec{a}, \vec{b}, \vec{c}]$ = $-[\vec{b}, \vec{a}, \vec{c}]$\\
Zyklisches Vertauschen der drei Vektoren ändert nichts:\\
$[\vec{a}, \vec{b}, \vec{c}] = [\vec{b}, \vec{c}, \vec{a}] = [\vec{c}, \vec{a}, \vec{b}]$\\
Multiplikation der Vektoren mit reellen Zahlen $\lambda$, $\mu$, $\nu$:\\
$[\lambda \vec{a}, \mu \vec{b}, \nu \vec{c}] = \lambda \mu \nu [\vec{a}, \vec{b}, \vec{c}]$\\
Addition zweier Vektoren\\
$[\vec{a} + \vec{b}, \vec{c}, \vec{d}] = [\vec{a}, \vec{c}, \vec{d}] + [\vec{b}, \vec{c}, \vec{d}]$

\section{Transformation: Translation in 2D}
Um eine Figur in eine Richtung zu verschieben, kann man alle ihre Punkte als Matrix zusammennehmen, diese in eine höhere Dimension nehmen (in homogene Koordinaten umschreiben) und die Translation in derselben Dimension vornehmen, indem man in der rechtesten Spalte der Einheitsmatrix die Verschiebungen vornimmt.

\subsection{Beispiel Translation 2D}

Gegeben sind die Punkte $A=(3,1)$, $B=(6,1)$ und $C=(5,4)$ und die Verschiebungsrichtung $\vec{t} = \begin{pmatrix}1 & 2
\end{pmatrix}$. 

Man nimmt nun die Punkte alle zusammen in einer Matrix und erweitert diese die homogenen Koordinaten mit dem Wert 1 in der dritten Dimension (dargestellt unterhalb des Strichs in der Matrix):

\[
Points = \begin{pmatrix}
A_x & B_x & C_x\\
A_y & B_y & C_y\\ \hline
1 & 1 & 1
\end{pmatrix} = \begin{pmatrix}
3 & 6 & 5\\
1 & 1 & 4\\ \hline
1 & 1 & 1
\end{pmatrix}
\]

Den Verschiebungsvektor macht man nun zur Matrix. Dazu wird die Einheitsmatrix der neuen homogenen Dimension genommen und als rechteste Spalte wird der Vektor $\vec{t}$ eingesetzt. $\vec{t}$ wird als zu 

$T=\begin{pmatrix}
1 & 0 & t_x\\
0 & 1 & t_y\\
0 & 0 & 1
\end{pmatrix}=\begin{pmatrix}
1 & 0 & 1\\
0 & 1 & 2\\
0 & 0 & 1
\end{pmatrix}$

Nun muss man nur noch $T * M$ berechnen:
\begin{align}
\begin{split}
T*Points &=\begin{pmatrix}
1 & 0 & 1\\
0 & 1 & 2\\
0 & 0 & 1
\end{pmatrix} * \begin{pmatrix}
3 & 6 & 5\\
1 & 1 & 4\\
1 & 1 & 1
\end{pmatrix}\\
&= \begin{pmatrix}
A_x' & B_x' & C_x'\\
A_y' & B_y' & C_y'\\ \hline
1 & 1 & 1
\end{pmatrix} = \begin{pmatrix}
4 & 7 & 6\\
3 & 3 & 6\\ \hline
1 & 1 & 1
\end{pmatrix}
\end{split}
\end{align}

Die Punkte der neuen Koordinaten sind also $A'=(4,3)$, $B'=(7,3)$ und $C'=(6,6)$.

\section{Transformation: Rotation um einen Winkel $\Phi$ in 2D}

Die Rotation um wird vorgenommen, indem folgende Matrix verwendet wird:
\[
T = \begin{pmatrix}
\cos (\Phi) & -\sin (\Phi) & 0\\
\sin (\Phi) & \cos (\Phi) & 0\\
0 & 0 & 1
\end{pmatrix}
\]

\section{Transformation: Rotation in 3D}

\subsection{Rotation einen Winkel $\Phi$ um die z-Achse}

$\textbf{R}_z(\Phi_z) = \begin{pmatrix}
\cos (\Phi) & -\sin (\Phi) & 0 & 0\\
\sin (\Phi) & \cos (\Phi) & 0 & 0\\
0 & 0 & 1 & 0\\
0 & 0 & 0 & 1
\end{pmatrix}$

Inverse davon ist:
$\textbf{R}_z^{-1}(\Phi_z) = \begin{pmatrix}
\cos (\Phi) & \sin (\Phi) & 0 & 0\\
-\sin (\Phi) & \cos (\Phi) & 0 & 0\\
0 & 0 & 1 & 0\\
0 & 0 & 0 & 1
\end{pmatrix}$

\subsection{Rotation einen Winkel $\Phi$ um die y-Achse}

$\textbf{R}_y(\Phi_y) = \begin{pmatrix}
\cos (\Phi) & 0 & \sin (\Phi) & 0\\
0 & 1 & 0 & 0\\
-\sin (\Phi) & 0 & \cos (\Phi) & 0\\
0 & 0 & 0 & 1
\end{pmatrix}$

Inverse davon ist:
$\textbf{R}_y^{-1}(\Phi_y) = \begin{pmatrix}
\cos (\Phi) & 0 & -\sin (\Phi) & 0\\
0 & 1 & 0 & 0\\
\sin (\Phi) & 0 & \cos (\Phi) & 0\\
0 & 0 & 0 & 1
\end{pmatrix}$

\subsection{Rotation einen Winkel $\Phi$ um die y-Achse}

$\textbf{R}_x(\Phi_x) = \begin{pmatrix}
1 & 1 & 0 & 0\\
0 & \cos (\Phi)  & -\sin (\Phi) & 0\\
0 & \sin (\Phi) & \cos (\Phi) & 0\\
0 & 0 & 0 & 1
\end{pmatrix}$

Inverse davon ist:
$\textbf{R}_x^{-1}(\Phi_x) = \begin{pmatrix}
1 & 1 & 0 & 0\\
0 & \cos (\Phi)  & \sin (\Phi) & 0\\
0 & -\sin (\Phi) & \cos (\Phi) & 0\\
0 & 0 & 0 & 1
\end{pmatrix}$

\subsection{Rotation um eine beliebige Achse}
\begin{enumerate}
\item Rotation um den Winkel $\Phi$ um die z-Achse (hier Matrix D genannt)
\item Rotation um den Winkel $\Theta$ um die frühere x-Achse mit der Matrix C.
\item Rotation um den Winkel $\Psi$ um die frühere z-Achse mit der Matrix B.
\end{enumerate}

Man hat nun die Matrizen:
\begin{figure}[!ht]
	\centering
	\includegraphics[width=0.7\linewidth]{fig/Rotation_um_bel_Achse.png}
	\caption{Rotationsmatrizen}
	\label{fig:spatprodukt}
\end{figure}

Diese kann man als eine Transformation zusammennehmen durch Multiplikation:
\begin{figure}[!ht]
	\centering
	\includegraphics[width=0.7\linewidth]{fig/Zusammengesetzte_Rotationsmatrix.png}
	\caption{Zusammengesetzte Rotationsmatrix}
	\label{fig:spatprodukt}
\end{figure}

\newpage
\section{Vektorprodukt}

\begin{figure}[!ht]
	\centering
	\includegraphics[width=0.7\linewidth]{fig/vektorprodukt_definition}
	\caption{Definition vom Vektorprodukt}
	\label{fig:vektorprodukt_definition}
\end{figure}

\begin{figure}[!ht]
	\centering
	\includegraphics[width=0.7\linewidth]{fig/vektorprodukt_example1}
	\caption{Vektorprodukt Example 1}
	\label{fig:vektorprodukt_example1}
\end{figure}

\begin{figure}[!ht]
	\centering
	\includegraphics[width=0.7\linewidth]{fig/vektorprodukt_example2}
	\caption{Vektorprodukt Example 2}
	\label{fig:vektorprodukt_example2}
\end{figure}

\subsection{Rechenregeln für das Vektorprodukt}

\begin{math}
	\vec{a} \times \vec{b} = -\vec{b} \times \vec{a}
\end{math}
Anti-Kommutativgesetz\\
\begin{math}
	\vec{a} \times (\vec{b} + \vec{c}) = \vec{a} \times \vec{b} + \vec{a} \times \vec{c}
\end{math}
Distributivgesetz\\
\begin{math}
	\lambda (\vec{a} \times \vec{b})=
	(\lambda \vec{a}) \times \vec{b} = 
	\vec{a}  \times (\lambda \vec{b})
\end{math}

