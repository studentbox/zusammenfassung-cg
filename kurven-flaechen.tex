\chapter{Kurven und Flächen}

\section{Kurven in der Ebene}
Gar nicht so schwierig! Wir haben irgendein Intervall, also eine Zahlenreihe von a nach b. Mathematisch ausgedrückt wäre das dann
\begin{displaymath}
[a,b]\rightarrow\mathbb{R}
\end{displaymath}
All diese Zahlen in diesem Intervall setzen wir einfach in eine Funktion hinein und voila - eine Kurve. Jetzt gibt es eine implizite Art, eine Kurve darzustellen und eine explizite Art.
\subsection{Implizite Darstellung}
Am Beispiel eines Kreises wäre die implizite Darstellung:
\begin{displaymath}
x^2 + y^2 - r^2 = 0
\end{displaymath}
Das heisst in einer impliziten Darstellung können wir z.B. nicht einfach die Zahlen von a nach b einsetzen, sondern die Gleichung ist einfach für alle zutreffenden Punkte = 0. Wenn wir Glück haben, ist die implizite Darstellung umwandelbar in eine explizite.
\subsection{Explizite Darstellung}
Eigentlich einfach eine Funktion.
Für den oberen Halbkreis:
\begin{displaymath}
y = \sqrt{r^2-x^2}
\end{displaymath}
und den unteren Halbkreis:
\begin{displaymath}
y = -\sqrt{r^2-x^2}
\end{displaymath}
\subsection{Parameter Darstellung}
\(t\) bedeutet hier die \textit{Zeit} und geht von a nach b. Beispiel mit dem Kreis, wo die \textit{Zeit} von \(0\) bis \(2\pi\) geht:
\begin{displaymath}
X(t) = 
\begin{pmatrix}
x_1(t) \\
x_2(t)
\end{pmatrix} = 
\begin{pmatrix}
r*cos(t)\\
r*sin(t)
\end{pmatrix}
\end{displaymath}
Die Parameterdarstellung ist nicht unbedingt eindeutig, also auch z.B. 
\begin{displaymath}
X(t) = 
\begin{pmatrix}
r*cos(2t)\\
r*sin(2t)
\end{pmatrix}
\end{displaymath}
beschreibt einen Kreis, wobei hier die \textit{Zeit} von 0 bis \(\pi\) geht - also er wird einfach 'schneller' durchlaufen.


\section{Kurven im Raum}
Dasselbe, einfach in grün. Eh 3D. Man hat einfach anstatt einen 2D Vektor in der Parameterdarstellung dann einen 3D Vektor.

\subsection{Länge der Kurve im Raum}
Das lässt sich ganz einfach mit einem Integral der ersten Ableitung der Kurve berechnen.

\begin{displaymath}
L = \int_a^b |f'(t)|,\mathrm{d}t
\end{displaymath}

\section{Kurvendarstellung}
\subsection{Polynomiale Darstellung}
Sagen wir, wir haben eine Ausgangskurve, die wir irgendwie im Computer effizient darstellen möchten. In der Abbildung \ref{fig:kurven_interpolieren} sind diese durch die gestrichelten Linien dargestellt. 
\begin{figure}[!ht]
	\centering
	\includegraphics[width=0.5\linewidth]{fig/kurven_interpolieren}
	\caption{Möglichkeit der Darstellung von Kurven}
	\label{fig:kurven_interpolieren}
\end{figure}
Nehmen wir die naive Variante und versuchen, die Kurve durch ein einfaches Polyom darzustellen, also irgendwas in der Form von:
\begin{displaymath}
f(x) = a_0 + a_1x + a_2x^2 + \dots + a_nx^n
\end{displaymath}
Leider nützt uns diese Variante nicht wirklich. Denn die durchgezogenen Linien zeigen, dass diese nie wirklich an die echte Kurve herankommt - mit mehr definierten Punkten wird es sogar noch schlimmer - siehe der rechte Teil der Grafik. Wir müssen also etwas besseres haben.

\subsubsection{Beispielrechnung - Methode der unbestimmten Koeffizienten}
\begin{figure}[!ht]
	\centering
	\includegraphics[width=0.3\linewidth]{fig/unbestimmte_koeffizienten}
	\caption{Beispiel Kurve}
	\label{fig:unbestimmte_koeffizienten}
\end{figure}

Wir können also folgende Punkte aus der Kurve lesen: \((0,1)\), \((1,1)\), \((2,0)\), \((3,1)\). Wir können also für die Gleichung

\begin{displaymath}
f(x) = a_0 + a_1x + a_2x^2 + a_3x^3
\end{displaymath}
für den ersten Punkt \(x = 0, y = 1\) wäre die Gleichung also:

\begin{displaymath}
1 = a_0 + a_1*0 + a_2*0 + a_3*0 = a_0
\end{displaymath}

Wir können das auch in einer Matrix darstellen;

\begin{displaymath}
\begin{pmatrix}
	1 & 0 & 0 & 0 \\
	1 & 1 & 1 & 1 \\
	1 & 2 & 4 & 8 \\
	1 & 3 & 9 & 27\\
\end{pmatrix}
\begin{pmatrix}
a_0 \\
a_1 \\
a_2 \\
a_3
\end{pmatrix}
= 
\begin{pmatrix}
1 \\
1 \\
0 \\
1
\end{pmatrix}
\end{displaymath}
und lösen die dann auf mittels Maple oder so - weil von Hand ist ja uncool \& lahm.
Nactheil an dieser Methode ist, dass es eben lahm ist, man muss wenn sich ein Punkt ändert immer das komplette Gleichungssystem wieder lösen. Die Lösung wäre auf jeden Fall:
\begin{displaymath}
f(x) = 1 + \frac{3}{2}x - 2x^2 + \frac{1}{2}x^3
\end{displaymath}
\subsubsection{Beispielrechnung - Methode nach Lagrange}
Wollen wir keine Gleichungssysteme lösen, so lässt uns Lagrange von der Qual erlösen. Schlechter Reim - ist aber trotzdem so.

Nehmen wir wieder wie Kurve aus Abbildung \ref{fig:unbestimmte_koeffizienten}. Wir haben wieder die bekannten Punkte, wobei der erste Punkt dann \(x_0 = 0\) und \(f(x_0) = 1\)  wäre usw.

Wir definieren dann Gleichungen wie in Abbildung \ref{fig:lagrange_gleichungen}.
\begin{figure}[!ht]
	\centering
	\includegraphics[width=0.7\linewidth]{fig/lagrange_gleichungen}
	\caption{Lagrange Gleichungen}
	\label{fig:lagrange_gleichungen}
\end{figure}
Die Funktion \(L_0(x)\) hat dann die Eigenschaft, dass sie an der Stelle \(x_0 = 1\) ist und an allen anderen Stützstellen, also \(L_0(x_1) = L_0(x_2) = \dots = 0\)  ist. Die Funktion \(L_1(x)\) hat dieselbe Eigenschaft, einfach ist sie an der Stelle \(x_1 = 0\). Dasselbe für die weiteren Gleichungen. Jetzt können wir einfach unsere Gleichung aufstellen:

\begin{displaymath}
f(x) = L_0(x)f(x_0) + L_1(x)f(x_1) + L_2(x)f(x_2)+ L_3(x)f(x_3)
\end{displaymath}
Was dann auch schon die Lösung ist.

Erklärung dazu; Mit den aufgestellten \(L_n\) Gleichungen können wir ja einfach die Stützstellen beliebig 'verschieben' - was wir zum Schluss mit den Faktoren \(f(x_0)\) auch machen. (Anmk. Autor - evt kann das jemand noch besser erklären, ich schreib jetzt einfach mal weiter im Text).
\subsection{Methode nach Newton}
Kommt wahrscheinlich nicht - war ja auch nicht in den Übungen vorhanden.
\section{Approximierende Splines}
Wir definieren einfach irgendwelche Punkte im Raum und bestimmen jeweils die Funktion zwischen den Punkten - diese ist eine Polynomiale Funktion max. 3 Grades. Die Kurve muss dann nicht durch die Punkte gehen.

\begin{figure}[!ht]
	\centering
	\includegraphics[width=0.3\linewidth]{fig/splines}
	\caption{Splines}
	\label{fig:splines}
\end{figure}

\subsection{Lineare Interpolation - Lineare Bézier Splines}
Wir haben einfach eine Linie mit den Anfangs \(P_0\) und Endpunkten \(P_1\). Jetzt müssen wir diese wahnsinnig schwierig beschreiben. Hier ist immer \(t \leq t \leq 1\).
\begin{enumerate}
	\item \(P(t) = (1-t)P_0+t*P_1\)
	\item \(P(t) = (P_1 - P_0)*t + P_0\)
	\item \(P(t) = (P_1, P_0) \begin{pmatrix}
	-1 & 1 \\1 & 0
	\end{pmatrix}\begin{pmatrix}
	t \\ 1
	\end{pmatrix}\)
\end{enumerate}
\subsection{Quadratische Bézier Splines}
\begin{figure}[!ht]
	\centering
	\includegraphics[width=0.2\linewidth]{fig/quadratic_bezier_spline}
	\caption{Quadratische Bézier Splines}
	\label{fig:quadratic_bezier_spline}
\end{figure}
Quadratische Bézier Splines sind eigentlich dasselbe wie die linearen Bézier Splines - einfach wird zwischen den linearen Splines nochmals interpoliert - yo dawg I heard you like interpolieren. In der Abbildung \ref{fig:quadratic_bezier_spline} sieht man das schön. Grafisch kann man das dort auch super erklären - denn vom Punkt in der Mitte zwischen \(P_0\) und \(P_1\)  wird einfach eine Linie gezogen zum Punkt in der Mitte zwischen \(P_1\) und \(P_2\) und dann in der Mitte dieser Linie ist dann der mittlere Punkt der fertigen Kurve. Wir rechnen also die 2 linearen Repräsentationen der Linien zusammen und das geht so:

Man hat zwei lineare Bézier Splines, die so definiert sind:
\begin{displaymath}
P^1_0(t) = (1-t)P_0 + t*P_1
\end{displaymath}
\begin{displaymath}
P^1_1(t) = (1-t)P_1 + t*P_2
\end{displaymath}

Jetzt setzt man diese einfach in die normale Bézier Repräsentation ein - also eine Funktion in die Funktion einfügen.
\begin{displaymath}
P(t) = (1-t)P^1_0(t) = t*P^1_1(t)
\end{displaymath}
Dann muss mans nur noch auflösen et voila:
\begin{displaymath}
P(t) = (1-t)^2P_0 + 2(1-t)t*P_1+t^2P_2
\end{displaymath}
Das, meine Kinder, wäre der ganze Zauber. Tschüss \& bis zum nächsten Mal!

\subsection{Kubische Bézier Splines}
Hehe, jetzt dachtest Du schon du bist fertig. Aber es wird nicht schwieriger. Wir haben jetzt einfach 4 Kontrollpunkte (\(P_0, \dots P_3)\))statt 3, also haben wir alle Definitionen:
\begin{displaymath}
P^1_0(t) = (1-t)P_0 + t*P_1
\end{displaymath}
\begin{displaymath}
P^1_1(t) = (1-t)P_1 + t*P_2
\end{displaymath}
\begin{displaymath}
P^1_2(t) = (1-t)P_2 + t*P_3
\end{displaymath}
Jetzt müssen wir alle 4 Gleichungen zusammenfügen:
\begin{displaymath}
P^2_1(t) = (1-t)P^1_0(t) = t*P^1_1(t)
\end{displaymath}
\begin{displaymath}
P^2_2(t) = (1-t)P^1_1(t) = t*P^1_2(t)
\end{displaymath}
Und diese dann nochmals in tha mix schmeissen:
\begin{displaymath}
P(t) = (1-t)P^2_1(t) = t*P^2_2(t)
\end{displaymath}
Was dann wieder aufgelöst:
\begin{displaymath}
P(t) = (1-t)^3P_0+3(1-t)^2*t*P_1+3(1-t)^2P_2+t^3P_3
\end{displaymath}
ergibt. Dem aufmerksamen Leser wird nicht entgangen sein, dass diese Bézier Splines Eigenschaften vom Pascal'schen Dreieck haben.

\subsection{Bernsteinpolynome}
Nicht aufgeben! Kurz ein Schluck Wasser und weiter gehts.

