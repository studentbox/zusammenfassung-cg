\chapter{Codebeispiele}

\section{Vertexshader}
\subsection{2D}

\subsection{3D}
\begin{lstlisting}
attribute vec3 aVertexPosition;
attribute vec3 aVertexNormal;
attribute vec4 aVertexColor;

uniform mat4 uPerspectiveMatrix;
uniform mat4 uCameraMatrix;

uniform mat4 uTransformationMatrix;
uniform mat3 uNormalMatrix;

varying vec4 vColor;
varying vec3 vLighting;

void main () {
    vColor = aVertexColor;
    vec4 position = vec4 ( aVertexPosition, 1.0);

    vec3 ambientLight = vec3(0.5, 0.5, 0.5);
    vec3 directionalLightColor = vec3(1, 1, 1);
    vec3 directionalVector = vec3(0, 1, 0);
    vec3 transformedNormal = normalize(uNormalMatrix * aVertexNormal);
    
    float directional = max(dot(transformedNormal, directionalVector),0.0);
    vLighting = ambientLight + (directionalLightColor * directional);

    gl_Position = uPerspectiveMatrix*uCameraMatrix*uTransformationMatrix*position;
}
\end{lstlisting}

\section{Fragmentshader}
\subsection{2D}

\subsection{3D}
\begin{lstlisting}
precision mediump float;

varying vec4 vColor;
varying vec3 vLighting;

void main () {
        gl_FragColor = vec4(vColor.rgb * vLighting, vColor.a);
}
\end{lstlisting}


\section{Setup-Methode}
\begin{lstlisting}
function setupAttributes(shaderProgram) {
    // finds the index of the variable in the program
    aVertexPositionId = gl.getAttribLocation(shaderProgram, "aVertexPosition");
    gl.enableVertexAttribArray(aVertexPositionId);

    aVertexNormalId = gl.getAttribLocation(shaderProgram, "aVertexNormal");
    gl.enableVertexAttribArray(aVertexNormalId);

    aVertexColorId = gl.getAttribLocation(shaderProgram, "aVertexColor");
    gl.enableVertexAttribArray(aVertexColorId);

    uTransformationMatrixId = gl.getUniformLocation(shaderProgram, "uTransformationMatrix");
    uPerspectiveMatrixId = gl.getUniformLocation(shaderProgram, "uPerspectiveMatrix");
    uCameraMatrixId = gl.getUniformLocation(shaderProgram, "uCameraMatrix");
    uNormalMatrixId = gl.getUniformLocation(shaderProgram, "uNormalMatrix");
}
\end{lstlisting}


\section{Startup-Methode}
\begin{lstlisting}
function startup() {
    canvas = document.getElementById("gameCanvas");
    gl = createGLContext(canvas);

    setupAttributes(loadAndCompileShaders(gl, "VertexShader.vert", "FragmentShader.frag"));

    gl.frontFace(gl.CCW); // defines how the front face is drawn
    gl.cullFace(gl.BACK); // defines which face should be culled
    gl.enable(gl.CULL_FACE);
    gl.enable(gl.DEPTH_TEST);


    //Background Color
    gl.clearColor(0.0, 0.0, 0.0, 1.0);

    perspective = mat4.create();
    mat4.frustum(perspective, -1.02, 1.02, -1.02, 1.02, 0.2, 100);
    gl.uniformMatrix4fv(uPerspectiveMatrixId, false, perspective);

    camera = mat4.create();
    mat4.lookAt(camera, [0, 0, 1.2], [0, 0, 0], [0, 1, 0]);
    gl.uniformMatrix4fv(uCameraMatrixId, false, camera);
    
	window.requestAnimationFrame(drawAnimated);
}
\end{lstlisting}


\section{Draw-Methode}
\begin{lstlisting}
function draw() {
    gl.clear(gl.COLOR_BUFFER_BIT | gl.DEPTH_BUFFER_BIT);
    ...
}
\end{lstlisting}